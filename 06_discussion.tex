We use novel administrative data to measure the association between owning a home in early adulthood and later-life longevity in the U.S., finding home owners live nearly 9 months longer than their renter peers. This national-level association is both demographically and statistically significant, and the magnitude of the association aligns closely with findings from past research findings~\citep{laaksonen_home_2008} in other contexts. 

In our second analysis, we use a sibling-based identification strategy to estimate the causal effect of home ownership on longevity, finding owning a home has a causal effect of over half a year on longevity. While we are making several assumptions to imbue our estimates with a causal interpretation, the robustness of our results and compellingness of plausible causal pathways suggest that home ownership likely has a causal effect on longevity.

These findings have implications for both explaining the Black-White mortality gap and for our understanding of the determinants of mortality more broadly. If home ownership in early adulthood has a causal effect on longevity, disparities in home ownership will have significant implications for our understanding of the determinants of mortality. While investigating the relative contribution of each causal pathway is outside of the scope of this paper, we speculate the some combination of stronger social networks, financial benefits, and lower health risks due to substandard housing are responsible for the causal effect of home ownership on mortality.  