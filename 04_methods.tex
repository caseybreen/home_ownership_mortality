For our preliminary analysis, we fit linear regression models on age at death. We include birth-year fixed effects to control for the different distribution of birth years across population subgroups. Because the CenSoc-DMF only contains deaths for the doubly-truncated window of 1975-2005, the magnitude of the reported differences across groups will be smaller than if we had the complete window of deaths. We fit a baseline model of the form: 

\begin{equation}
    D_i = \beta_0 + \lambda_t + \delta_{own}
\end{equation}

where $D_i$ is age of death, $\beta_0$ is the general intercept, $\lambda_t$ is a fixed effect for a given year of birth $t$, and $\delta_{own}$ is a dummy variable for whether an individual is a renter or a home owner. Our full model adjust for sociodemographic characteristics and includes family fixed effects. 

\begin{equation}
    D_i = \beta_0 + \lambda_t + \delta_{own} + Z_i\boldsymbol{\beta} + \Omega_{family}
\end{equation}

$Z_i\boldsymbol{\beta}$ is the set of control variables and $\Omega_{family}$ is a family fixed effect. To adjust for the truncation bias, an inflation factor of 1.3 is necessary for all effect sizes \citep{goldstein_censoc_2021}.




