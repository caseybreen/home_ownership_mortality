With parametric assumptions, maximum likelihood methods enable
estimation of age-specific mortality with multivariate predictors,
adapting the usual method of parametric survival analysis to our
specific case of observing only those individuals who die. The
likelihood associated with a set of observed ages of death $x_i$ with
parameters $\theta$ (e.g., the intercept and slope of the log-Gompertz
curve, which may themselves be functions of covariates) is given by
the product of the normalized densities, with truncation on the right
at age $x_r$ and on the left at age $x_l$:
\begin{equation}\label{trunc_mle_eq}
L (\theta) = \prod_i L_i (\theta) = \prod_i \frac{f(x_i | \theta )}{
F(x_r | \theta ) - F(x_l | \theta)},
\end{equation}
where $f$ is the density and $F$ is the cumulative distribution.

For example, a proportional hazards model for the effect of covariates on mortality takes the following form. The hazard of an
individual $i$ aged $x$ with covariates $Z_i$ is given by


\begin{equation}\label{ph_educ_eq}
h_i(x | \beta) = h_0(x) e^{\boldsymbol{\beta} Z_i},
\end{equation}
with a baseline hazard schedule over age $x$ of $h_0(x)$.
If the baseline is Gompertz,
$$
h_i(x | \theta_Z) = a_0 e^{b_0 x} e^{\boldsymbol{\beta} Z_i},
$$
where $a_0$ and $b_0$ are baseline Gompertz parameters. Alternatively,
we can write
$$
h_i(x | \theta_Z) = a_i e^{b_0 x},
$$
where $a_i$ is the individual level Gompertz intercept, modeled as $
a_0 e^{(\boldsymbol{\beta} Z_i)}$.

In this case, the observed data would contain for each person values
$x_i$ for the age of death, $Z_i$ for covariates (e.g., years of education, place of birth), and the right and left truncation ages $x_r$ and $x_l$ for each cohort. The model estimates would be the parameter values
$\hat{a_0}, \hat{b_0}$ and $\hat{\beta}$.

The approach of explicitly including truncation ages has several
advantages. First, it is fairly easy to understand and is
computationally flexible. One can use standard optimization routines
to maximize the likelihood. Or, one can rely on analogous Bayesian
methods. Both approaches provide uncertainty estimates, with the
optimization method basing uncertainty on the curvature of the
likelihood surface, an the Bayesian approach providing a posterior
distribution.

Second, the functional form of the covariates (and of the baseline
hazard) is flexible. We use the Gompertz form here with the
proportional hazards assumption (estimating one common Gompertz slope
$b_0$ for all cohorts and sub-groups). But it is possible to fit
hierarchical models that allow for arbitrary structure in how
parameters co-vary. For example, one might want to allow the Gompertz
slope to vary (slowly) over time and differ between sub-groups.

Third, it is possible to use other parametric forms besides the
Gompertz, incorporating mortality deceleration at very old ages, a
Makeham constant. It is also possible, as \citet{alexander_deaths_2018} has shown,
to estimate the principal components of known age-schedules (e.g.,
from HMD) and penalize estimates that differ from these schedules as
described by linear combinations of these components.

Model estimation can be challenging, requiring correct specification
of the likelihood function in computer code, as well as the usual
challenges of obtaining convergence with optimization methods,
including the specification of starting values. Bayesian estimation is
promising and flexible but can be slow when working with the
many thousands or even millions of cases in administrative data.

A specific challenge working with data like CenSoc is that we would
like to have models that remove bias in a consistent way across
cohorts with different age windows of observation. This may not be the
case if, for example, the assumed baseline hazard such as Gompertz is
more appropriate for some ages than for others. Any estimates studying
change over time from one birth cohort to the next need to wrestle
with the possibility that changing parameters estimates could result
from real changes in effects over time, from artifacts resulting from
variation in departures from model assumptions, or from a combination
of these.

\begin{figure}[h]
\centering 
	\includegraphics[width = 1\textwidth]{figs/histograms_and_hazards.pdf}
	\caption{For CenSoc-DMF cohort of 1910: panel (a) shows a histogram of age of death for those who completed high school; the blue dashed curve shows our Gompertz-model based estimates of deaths. Panel (b) shows a histogram of age of death for those who did not complete high school; the dashed red line shows our Gompertz-model based estimates number of deaths. Panel (c) plots the model-based hazard ratio estimates (solid lines) vs. the observed hazard ratios (dashed lines) for those who have completed high school (blue) and those who did not (red). \todo{fix lines}}
	\label{fig:hist}
\end{figure}


\subsection{Comparison with Regression coefficients}
