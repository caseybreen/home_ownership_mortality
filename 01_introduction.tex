In the United States, owning a home has been linked to positive health and mortality outcomes~\citep{rolfe_housing_2020}. However, less is known about whether this relationship is causal -- is there something inherent about owning a home ownership that causes people to live longer lives? Or is the observed association driven entirely by unmeasured shared confounders, such as family wealth or social capital? 

Home ownership promotes stronger social networks, community-based investment in cultural capital, and serves as an important vessel for the inter-generational accumulation of wealth, all of which have strong protective effects on mortality. Home ownership is also associated with lower health risks from substandard housing: infectious disease, chronic disease, and lower mental health. 

Understanding the mortality consequences of home ownership has scientific and policy implications. The magnitude of the association between home ownership and longevity is large: a comprehensive study of Finnish homeowners found that owning a home -- as opposed to renting -- attenuates excess mortality by over 30\%, after adjusting for basic socioeconomic variables~\citep{laaksonen_home_2008}. If home ownership has a causal effect on longevity, expanding policies that allow historically marginalized groups to own and keep a home may serve as a cost-effective strategy to mitigate the profound mortality disparities in the US. However, if home ownership by itself does not increase longevity, such interventions would be of little value from a public health perspective.  

We use linked 1940 Census and mortality data to produce the first US-based estimates of the relationship between home ownership and longevity. For our main analysis, we use a group of over 2 million men born in the US between 1905 and 1915. The large sample size allows us to definitively quantify the association between home ownership and longevity. We then use a representative panel of 50,000 male siblings, drawn from Census and mortality records, to adjust for hard-to-measure early life characteristics, such a shared family environment and wealth. We find siblings who own homes live longer than their siblings who rent, adjusting for education, occupation, and urbanicity. Taken together, these results suggest that home ownership may have a causal effect on longevity. 
