\subsection{Causal Pathways} 

A handful of studies have suggested that owning a home in the U.S. is associated with better health and mortality outcomes. While there has been no causal evidence that home ownership is associated with longer longevity, there are several compelling theoretical pathways suggesting that this relationship is causal. 

\begin{figure}[H]
  \centering
  \includegraphics[width=1\textwidth]{figs/dag_homeownership.pdf}
  \caption{Causal pathways between home ownership and mortality.}
  \label{fig:causal_pathways}
\end{figure}

First, homeowners stay longer in a unit than their renter counterparts~\citep{rohe_homeownership_1996}. This promotes promotes stronger social networks and community, a known determinant of lower mortality. Further, these stronger social networks are associated with higher social capital, among low and moderate income households. A large body of work has connected social network and social support to mortality outcomes~\citep{cobb_social_1976, berkman_emotional_1992, berkman_social_1979, smith_social_2008}. We also hypothesize that homeownership can facilitate the investment in human capital relevant to health, such as stronger relationships with primary physicians. 

Second, home ownership is a key vessel for wealth accumulation in the U.S. during this time. While there is substantial heterogeneity in the financial returns of home ownership by timing of purchase, location of purchase, and length of ownership, there are several ways in which home ownership helps accumulate wealth. First, homes themselves appreciate in value. Second, homeowners enjoy tax benefits in the form of tax deductions of on mortgage interest and no capital gains taxes. Finally, home ownership plays an important role in retirement savings, as homeowners don't access equity directly rather the rent-free use of the property. Home ownershio was historically, and largely remains, the single largest component of non-pension wealth in the United States. 

Third, homeowners are at lower risk than renters of health issues brought about from substandard housing. The quality of housing for renters is generally considered to be lower than for ownership. Substandard housing effects mortality through infectious disease, such as lack of clean drinking water, higher exposure to insects and rats, and overcrowding, all factors that can lead to higher risk of infectious disease mortality. Additionally, there is some evidence that damp, cold conditions can lead to lower mental health. 

Fourth, owning a home is associated with general feelings of security and stability. This can reduce stress, a known determinant of lower health and mortality. 

\subsection{Racial Differences in Home Ownership} 

In this study, we observe home ownership status (own vs. rent) in 1940, when the men in our sample are between 25 and 35 years of age. \cref{fig:home_ownership_rate_1940_bw} gives a sense both of the considerable variation in home ownership by age in 1940, and the large Black-White inequality in home ownership rates. 

\begin{figure}[H]
  \centering
  \includegraphics[width=1\textwidth]{figs/home_ownership_rate_1940_bw.png}
  \caption{Black-White differences in home ownership rates in 1940.}
  \label{fig:home_ownership_rate_1940_bw}
\end{figure}